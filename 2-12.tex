\section{Some basic algebraic geometry}

\textbf{\blu{2-12: missed class today. Notes are sketchy.}}

Let $k$ be a field of characteristic $\ne 2,3$. Let $C/k$ be a ``nice" curve of genus $g$. $\Pic^0(C)$, the Jacobian, is an abelian variety of dimension $g$, defined over $k$. %\Pic n(C) (arrow to itself)
We have that $C\to \Pic^{\wedge}(C)$ is injective if $g\ge 1$, an isomorphism if $g=1$. 
For $D\in \Pic^{n}(C)(k)$, indices a map
\bal
\pi_D:C&\to \Pic^0(C)\\
P&\mapsto n(P)-D.
\end{align*}

For the case $g=1$, $C\cong \Pic^1(C)$ is an $E$-torsor. For $D\in \Pic^n(C)(k)$, 
\[
\pi_P(\ub c{\in C}+\ub e{\in E}) = \pi_P(c)+ne.
\]

\begin{df}
An \vocab{$n$-cover of $E$} is a torsor $C$ of $E$ along with a morphism $\pi:C\to E$ defined over $k$ such that for all $c\in C(\ol k), e\in E(\ol k)$, 
\[
\pi(c+e)=\pi(c)+ne.
\]
\end{df}
Note that $\pi$ is determined by (the divisor class of $E=\Pic^0(C)$) $\pi(P)$ for any fixed $P\in C(\ol k)$. 
If $\pi:C\to E=\Pic^0(C)$ is an $n$-cover, then the divisor class $D=n(P)-\pi(P)\in Pic^n(C)(\ol k)$ doesn't depend on $P\in C(\ol k)$ and is thus Galois invariant, i.e., $n(P)-\pi(P) \in \Pic^n(C)(k)$. Then $\pi=\pi_0$. So we have the correspondence between divisors $D\in \Pic^n(C)(k)$ and $n$-covers $C\to \Jac(C)$.

\begin{ex}
The trivial $n$-cover is multiplication by $n$, $[n]:E\to E$. For any $d\in E(k)$, 
\bal
[n]_d:E&\to E\\
P&\mapsto nP+d
\end{align*}
is a $n$-cover equivalent to $[n]$ if $d\in nE(k)$.
\end{ex}

\begin{df}
Two $n$-covers $\pi_1:C_1\to E, \pi_2:C_2\to E$ are equivalent if there exists an isometry $\al:C_1\xrc C_2$ of $n$-torsors ($\al(c_1+e) = \al(c_1)+e$) such that the following commutes:
\[
\ctr{C_1}{C_2}{E}{\al\cong}{}{}{\pi_1}{}{\pi_2}
\]
\end{df}
\begin{ex}
$[n]_{d_1}$ and $[n]_{d_2}$ are equivalent iff $d_1\equiv d_2\pmod nE(k)$. We obtain an inclusion
\[
E(k)/nE(k) \hra \{\text{$n$-covers of $E$}\}/\text{equivalence}
\]
\end{ex}
\begin{ex}
Suppose $\pi:C\to E$ is an $n$-cover. Then $\pi$ is equivalent to $[n]$ iff there exists $P\in C(k)$ with $\pi(P)=O$.
\end{ex}
\begin{ex}
Say $\pi$ is \ivocab{soluble} if $C(k)\ne \phi$, so $C\cong E$ as $E$-torsors (the isomorphism sending $P$ to $O$). Then $\pi$ is equivalent to $[n]_d:E\to E$ for some $d\in E(k)/nE(k)$, $d=\pi(P)$. We have the bijection 
\[
E(k)/nE(k) \hra \{\text{soluble $n$-covers of $E$}\}/\text{equivalence}
\]
\end{ex}
\begin{ex} 
If $k$ is algebraically closed, every $n$-cover is soluble and $E(k)/nE(k)=0$. Any $n$-cover $\pi:C\to E$ is equivalent to $[n]$. Thus, over arbitrary $k$, $n$-covers modulo equivalence are twists of $[n]:E\to E$. 
\bal
\{\text{$n$-covers}\}/\text{equivalence}&\cong H^1(k, \Aut([n]:E\to E))\\
&= H^1(k, E[n]).
\end{align*}
\end{ex}
\begin{df}
Let $k$ be a global field. A $n$-cover $\pi:C\to E$ is \ivocab{locally soluble} if for every place $v$ of $k$, $C(k_v)\ne \phi$. 
\end{df}
We have
\bal
\Sel_n(E/k):&=\{\text{locally soluble $n$-covers of $E$}\}/\text{equivalence}\\
&=\{\text{elements of $H^1(k,E[n])$ lying locally in the image of $E(k_v)/nE(k_v)$}\}.
\end{align*}
There is a map 
\[
E(k)/nE(k) \hra \Sel_n(E/k)
\]
so 
\[
|\Sel_n(E/k)|\ge |E(k)/nE(k)|\ge n^{\rank E(k)}.
\]
This means we can bound average ranks in terms of average sizes of Selmer groups.

Our goal is to write down equations for locally soluble $n$-covers and count solutions.

A locally soluble $n$-cover corresponds to $D\in \Pic^n(C)(k)$. (The $n$-cover can be represented by a rational divisor since the $n$-cover is locally soluble.) This corresponds to a map $C\to \Pj^{n-1}$. 
\begin{ex}
For $n=2$, the map $C\to \Pj^1$ corresponds to writing $C:z^2 = f(x,y) = ax^4+bx^3y+cx^2y^2+dxy^3+ey^4$ where $f$ is unique up to $\PGL_2$-equivalence. ($C\sub \Pj(1,1,2)$).

For example, $D=(1,0,\sqrt a) + (1,0,-\sqrt a)$ on $E:y^2=x^3-\fc I3x - \fc J{27}$ where 
\bal
I&=12ae - 3bd + c^2\\
J&=72 ace + 9bcd - 27 ad^3 - 27 eb^3 - 2c^3
\end{align*}
are the $\PGL_2$-invariants of binary quartic forms. $\pi:C\to E$ is trivial iff $f(x,y)$ has a linear factor over $K$. Call a 2-cover \vocab{irreducible} if $f$ has no linear factor over $k$.

(Note that $[2]:E\to E$ corresponds to $f(x,y)=x^3y - \fc I3 xy^3 - \fc J{27}y^4$.) 
\end{ex}
The group $G_2:=\PGL_2$ acts on $V_2=\Sym^4(2)$, the set of binary quartic forms. The action is given by 
\[
\ga f(x,y) = \rc{(\det \ga)^2} f((x,y)\ga).
\]
The invariants $[V_2]^{\si_2}$ are the free polynomial ring generated by $I,J$.

\begin{ex}
For $n=3$, $C\to \Pj^2$ given by $D$, write
\[
C:ax^3+bx^2y^2 + cx^2z + \cdots +jz^3=0.
\]
The terms $I,J$ in the equation for $E$ are defined using the Hessian. 

$\pi:C\to E$ is trivial iff $C$ has a rational flex point (triple tangent). Otherwise, call $\pi$ \vocab{irreducible}.

The group $G_3:=\PGL_3$ acts on $V_3 = \Sym^3(3)$ by 
\[
\ga f(x,y,z)=\rc{\det\ga}f((x,y,z)\ga).
\]
\end{ex}
\begin{ex}
For $n=4$, $C\to \Pj^3$, $C:Q_1(t_1,\ldots, t_4)=Q_2(t_1,\ldots, t_4)=0$, $C':z^2=\det(xQ_1-yQ_2)$ is a 2-cover of $\Jac(C)$. $E:y^2=x^3-\fc I3 - \fc J{27}$. For fixed $x,y$, $xQ_1-yQ_2$ is a quadric surface in $\Pj^3$ and a solution $z$ to $z^2 = \det(xQ_1-yQ_2)$ corresponds to a ruling of the surface. 

For $p\in C=((Q_1=0)\cap (Q_2=0))$, there is a unique $(x,y)$ such that the line $T_pC$ lies in $xQ_1-yQ_2=0$. The ruling containing this line corresponds to a point $(x,y,z)\in C^1$. We obtain a map $C\to C^1$. We call a 4-cover $C\to E$ irreducible iff $\det(xQ_1-yQ_2)$ has no linear factors. $[4]:E\to E$ corresponds to
\bal
Q_1&=
\left[\begin{array}{cccc}
 &  &  & 1\\
0 & 0 & 0 & 0\\
 &  & 1\\
1
\end{array}\right]
&
Q_2&=\left[\begin{array}{cccc}
 &  & -1\\
 & -1 &  & -\frac{I}{6}\\
-1\\
 &  & -\frac{I}{6} & -\frac{J}{27}
\end{array}\right]
\end{align*}
The group 
\[
G_4 = \set{(g_2,g_4)\in \GL_2\times \GL_4}{\det (g_2)\det(g_4)=1}/\set{(\la^{-2},\la)}{\la\in \G_m}
\]
acts on $V_4 = 2\ot \Sym^2(4)$.
\end{ex}
\begin{ex}
For $n=5$, 
\[
G_5=\set{(g_1,g_2)\in \GL_5\times \GL_5}{(\det g_1)^2 \det g_2=1}/\{(\la, \la^{-2})\}
\]
acts on $V_5 = (\La^2 5)\ot 5$.
\end{ex}
We don't know what happens for $n\ge 6$.

\begin{thm}
Let $k$ be a field of characteristic $\ne 2,3$. Let $n=2,3,4,5$, $E:y^2 = x^3 - \fc I3 x - \fc J{27}$. 
\begin{enumerate}
\item
There is a bijection
\[
E(k)/nE(k) \lra \text{soluble $G_n(k)$-orbits in $V_n(k)$ with invariants $I,J$}
\]
\item
For $v\in V_n(k)$ with invariants $I,J$ there is an isomorphism 
\[
\Stab_{G_n}(V)\cong E[n]
\]
as group schemes.
(cf. Ho-Bhargava: regular representations of genus 1 curves; Birch-and-Swinnerton-Dyer: notes on elliptic curves 1, Cremona-Fischer-Stoll: minimalization and reduction $n=2,3,4$, explicit $n$-descent for elliptic curves I)
\item
Suppose now that $K$ is a global field of characteristic $\ne 2,3$. Then  there is a bijection
\[
\Sel_n(E/k) \lra \text{locally soluble $G_n(K)$-orbit of $V_n(K)$ with invariants $I,J$},
\]
\end{enumerate}•
\end{thm}
\begin{thm}
Let $K$ be a global field of characteristic $\ne 2,3$. When all elliptic curves over $K$ are ordered by height, the average size of $\Sel_n(E/K)$ for $n=2,3,4,5$ is 3, 4, 7, 6. (These are $\si(n)=\sum_{d\mid n}d$.) The average number of irreducible locally soluble $n$-covers mod equivalence is $n$ for $n=2,3,4,5$.
\end{thm}

An application: Using $5^r \ge 20r-15$ for $r\in \Z_{\ge 0}$, the average rank is $\le \fc{21}{20}=1.05$.

\subsection{Heights over global fields}

Let $K$ be a number field, and $M_\iy$ be the set of archimedean places. 

Let $K$ is the function field of a nice curve $X/\F_q$. Then  $M_\iy = S_0$ is the set of valuations at any fixed finite nonempty set of points on $X$. Let $\cO_K$ be the functions which are regular on $X\bs S_0$.

Two elliptic curves $E_{A,B},E_{A',B'}$ are isomorphic iff $A'=c^4A$, $B'=c^6B$ for some $c\in K^{\times}$. We can think of $E_{A,B}$ as a point $(A,B)\in \Pj(4,6)(K) = \A^2(K)/G_m(K)$.

Given $(A,B)\in S(K)$, define $\wt I = \set{\al\in K}{\al(A,B) \in S(\cO_K)}$ and
\[
H(A,B) = N\wt I \prod_{v\in M_{\iy}}\max\{|A|_v^{\rc 4}, |B|_v^{\rc 6}\}.
\]

The strategy of the proof of the theorem is as follows.
\begin{enumerate}
\item
Find a ``convenient" fundamental domain for the action of $G_m(K)$ on $S(K)$, i.e., $\cal E\sub S(\cO_K)$. $\cal E$ is the set of elliptic curves up to isomorphism. (Homework)
\item Compute $\#\cal E$. (Simple case of step 3)
\item Compute the number of locally soluble orbits with  invariants in $\cal E$. (Find integral representations in rational orbits.)
\item
Divide, look for cancellation, and profit. (Product formula)
\end{enumerate}•

Question: start with $v\in V_n(K)$ locally soluble with integral invariants $(I,J)\in \cal E$. Does there exist $v'\in V_n(\cO_K)$ that is $G_n(K)$-equivalent to $V$?

\begin{thm}[Local minimization]
Suppose $V_\ph$ is a nonarchimedean local field with ring of integers $\cO_\ph$. Let $v\in V_n(K_\ph)$ be soluble with invariants $(I,J)\in 6S(\cO_K)$. Then $G_n(K_\ph)v\cap V_n(\cO_\ph)\ne \phi$.
\end{thm}
\begin{proof}
We have correspondences between
\begin{enumerate}
\item
soluble cubics,
\item
$d\in E(K_\ph)/nE(K_\ph)$,
\item
$[n]_d:E\to E$,
\item
$E\to \Pj^{n-1}$ via the divisor $(n-1)\iy+(P)$ where $P$ is in the class of $d$.
\end{enumerate}
The idea is to find a good representative $P$ for $d$. Write down $E\to \Pj^{n-1}$. Observe they are integral.

If $d=0$, $x^3y-\fc I3 y^3 - \fc{J}{27} y^4\in \Sym^4(2)$.

If $d\ne 0$, let $P=(a,b)$ be an arbitrary representative. If $a,b\in \cO_\ph$, then the explicit embedding is integral (do the change of variables $(a,b)\lra (0,0)$ and write down sections). 
Otherwise, $P\in nE(K_\ph)$.

For $n=2$, $L=K_\ph[x]/(x^3+Ax+b)=K_\ph[\te]$. We have a map
\bal
E(K_\ph)/2E(K_\ph) & \hra (L^{\times}/L^{\times 2})_{\text{Norm}=1}\\
(a,b)&\mapsto (a-\te)\in L^{\times 2}
\end{align*}
since $(a,b)\in 2E(K_\ph)$.
\end{proof}

\subsection{Patching}
Suppose $v\in U_n(K)$ is locally soluble with invariants in $\Si$. Then there exist $g_\ph\in \si_n(K_\ph)$ such that $g_\ph v\in V_n(\cO_\ph)$ for all $\ph\nin M_{\iy}$. Consider the adele $(g_\ph)_\ph\in \si_n(\A_f)$.
\[
d_{\si_n,K}:=\pa{\prod_{\ph\nin M_{\iy}}G_n(\cO_\ph)}\bs G_n(\A_f)/G_n(K)
\]
is always finite (and $=1$ for $K=\Q$), by Borel, Prasad, and Conrad.

Fix lifts of $cl_{G_{n,K}}$ in $G_n(\A_f)$. Then there exist $(g_\ph')\in \prod_{\ph\nin M_{\iy}} G_n(\cO_\ph)$, $\be\in cl_{G_{n,\A_f}}$ one of the chosen lifts, $h\in G_n(K)$, such that $(g_\ph)=(g_\ph')\be h$. Then $hv\in V_n(K)\cap \be^{-1}\pa{\prod_{\ph\nin M_{\iy}V_n(\cO_\ph)}}=V_{n,\be}$ where $V_{n,\be}$ is commensurable with $V_{n,1}=V_(\cO_K)$. The subgroup
\[
G_{n,\be} = G_n(K) \cap \be^{-1}\pa{\prod_{\ph\nin M^{\iy}}G_n(\cO_\ph)}\be
\]
is commensurable with $G_{n,1}=G_n(\cO_\ph)$ and acts on $V_{n,\be}$. Count $\#G_n(K)\bs V_n^{\text{loc. sol.}}(K)$ by counting $\#G_{n,\be}\bs V_{n,\be}$. 

Some problems:
\begin{enumerate}
\item
One $v$ can correspond to multiple $\be$.
\item
One $G_n(K)$-orbit in $V_{n,\be}$ might break up into multiple $G_{n,\be}$-orbits.
\item 
Elements of $V_{n,\be}$ might not be locally soluble.
\end{enumerate}•
The solution is to count lattice points with a weight function.
